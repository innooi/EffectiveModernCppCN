\chapter{类型推导}
C++98有一套单一的类型推导规则:函数模板,C++11对此做了少量的修改,并加入了2种新的方法:一种是{\M auto},另一种是{\M decltype}。C++14则进一步扩展了{\M auto}和{\M decltype}能够使用的语境。类型推断的广泛应用将程序员从繁重的类型拼写工作中解放出来。在源码中某一处的修改能够自动的通过类型推导应用到其他的地方,提升了C++开发的弹性。然而,类型推导也会使得程序更加难以理解,因为编译器做出的类型推导很有可能与你希望的方式不同。

因此,如果不深入的理解类型推导的原理,高效地使用现代C++编程是不可能的。因为有太多的场景会用到它:调用函数模板,大多数{\M auto}出现的地方,{\M decltype}表达式中和C++14中神秘的的{\M decltype(auto)}构造应用的地方,

本章涵盖了每个C++开发者都应了解的类型推导知识,阐述了模板类型推导是如何工作、{\M auto}在此基础上如何建立规则和{\M decltype}如何按照自己独立的规则工作。其中甚至描述了如何强制令编译器推导的类型结果可见,使你能够确保编译器推导的结果是你所希望的。


\section{条款1:理解模板类型推导}

大多数复杂系统的用户都只关心系统所带来的功效,却并不关心它的工作原理。从这个层面来看,C++的模板推导是成功的。尽管大多数的程序员对类型推导的工作方式并不了解,但他们使用函数模版传递参数都获得了满意的结果。

如果你也是这个群体的一部分,我将告诉你一个好消息和一个坏消息。好消息是模板的类型推导是现代C++中最引人注目功能之一:{\M auto}的基石。如果你对于C++98中的{\M auto}很熟悉,那么C++11中的{\M auto}对你来说也会很熟悉。坏消息是当模板类型推断在有{\M auto}的上下文中使用时,相对于其在其他情况下(简单模板类型中)的应用变得更加令人疑惑了。因此,理解{\M auto}所依赖的模板类型推导的原理是非常重要的。本条款涵盖了该方面你所需要了解的所有知识。

如果你愿意忽略少量的伪代码,我们可以思考下面一个函数模板的例子:

\begin{lstlisting}
template<typename T>
void f(ParamType param);
\end{lstlisting}

函数的调用形如:

\begin{lstlisting}
f(expr); //使用表达式调用f函数
\end{lstlisting}

在编译过程中,编译器使用\textit{expr}推导2个类型:一个是{\M T}另一个是\textit{ParamType}。这2个类型往往是不同的,因为\textit{ParamType}通常都会包含修饰符,如{\M const}或者是引用限定符。例如,如果模板是这样声明的:

\begin{lstlisting}
template<typename T>
void f(const T& param); //ParamType是const T&类型
\end{lstlisting}

并且我们这样调用:

\begin{lstlisting}
int x = 0;

f(x); //调用f,实参是int类型
\end{lstlisting}

{\M T}被推导为{\M int}类型,但\textit{ParamType}被推导为const int\&。

人们通常希望{\M T}的类型推导和传入函数的参数类型是一致的,例如{\M T}的类型与\textit{expr}是相同的。在上述的例子中,就是这种情况:{\M x}是{\M int}类型,{\M T}被推导为{\M int}类型。但并不总是这样,{\M T}的类型推导不仅仅和\textit{expr}的类型有关,也和\textit{ParamType}的类型相关。下面有3个例子:

\begin{itemize}
\item{\textit{ParamType}是一个指针或者引用,但不是一个通用引用(Universal Reference)。(通用引用在条款24中描述,在这里,你只需要知道它存在,并且和左值引用、右值引用不同。)}
\item{\textit{ParamType}是一个通用引用。}
\item{\textit{ParamType}既不是指针,也不是引用。}
\end{itemize}

接下来我们会介绍3种类型推导的情形,每一个调用都会以我们的通用模板为基础:

\begin{lstlisting}
template<typename T>
void f(ParamType param);

f(expr); //从expr中推导T和ParamType的类型
\end{lstlisting}

\subsection*{情形1:\textit{ParamType}是一个指针或者引用,但不是一个通用引用}
在这种情况下,类型推导是这样工作的:

1. 如果\textit{expr}的类型是一个引用,忽略引用的符号。

2. 然后模式匹配expr的类型来决定ParamType的类型从而决定T的类型

例如,如果这是我们的模板:
\begin{lstlisting}
template<typename T>
void f(T& param); //param是一个引用
\end{lstlisting}
然后我们有如下变量声明:
\begin{lstlisting}
int x = 27; //x是int类型
const int cx = x; //cx是const int类型
const int& rx = x; //rx是x的常量引用
\end{lstlisting}

则param和T的类型推导如下:
\begin{lstlisting}
f(x); //T是int,param的类型是int &

f(cx); //T是const int,param的类型是const int&

f(rx); //T是const int,param的类型是const int &
\end{lstlisting}

在第二个和第三个调用中,注意由于cx和rx是常量类型,T被推导为const int,因此产生了const int\&类型的形参。这对于调用者来说非常重要。当他们传递一个const对象给一个引用形参时,他们希望这个对象仍是不可修改的,例如:形参是对常量的引用。这也是为什么传递一个const对象给一个使用T\&作为形参的模板是安全的:对象的常量性(constness)被推导为了T类型的一部分。

在第三个例子中,注意尽管rx的类型是一个引用,T仍被推导为了一个非引用类型。这是因为rx的引用性(reference-ness)在类型推导的过程中被忽略了。

这些例子中使用的都是左值引用形参,类型推导同样以相同的方式能够作用于右值引用形参。当然,只有右值的实参可能会被传递给右值引用的形参,但是这对类型推断没有什么影响。

如果我们把f的形参由T\&改为const T\&,就会发生一些变化,但也并不令人惊讶。cx和rx的常量性依然被保留,但是因为我们已经假定param的是一个对常量的引用,所以const不再是T类型的一部分了。
\begin{lstlisting}
template<typename T>
void f(const T& param); //param现在是一个常量引用

int x = 27;
const int cx = x;
const int &rx = x;

f(x); //T是int,param的类型是const T&

f(cx); //T是int,param的类型是const T&

f(rx); //T是int,param的类型是const T&
\end{lstlisting}
同样,rx的引用性在类型推导的过程中被忽略。

如果param是一个指针(或是指向const的指针)而不是一个引用,规则并没有本质上改变:

\begin{lstlisting}
template<typename T>
void f(T* param); //param现在是一个指针

int x = 27;
const int *px = &x; //px是一个将x视作const int的指针

f(&x); //T是int,param的类型是int *
f(px); //T是const int,param的类型是const int *
\end{lstlisting}
现在,你可能发现你自己在不断的打哈欠和点头,因为C++的类型推导在引用和指针形参上工作得如此自然。每件事情都是如此清晰,与你希望的类型推导系统完全一样。

\subsection*{情形2:\textit{ParamType}是一个通用引用}
当使用了通用引用后,事情就变得不是那么清楚了。形参似乎被声明为一个一个右值引用(例如:函数模板中的形参被声明为T\&\&),但是它们在传入一个左值实参后的表现却完全不同。完整的故事会在条款24中描述,这里有一个概要的版本。

\begin{itemize}
\item{如果expr是一个左值表达式,那么T和ParamType都被会推导为左值引用。这是非常不同寻常的。首先这是唯一一种情形:T被推导为一个引用。第二,尽管ParamType被声明为一个右值引用,但是其推导的类型却是一个左值引用。}
\item{如果expr是一个右值表达式,则按照情形1的方式处理。}
\end{itemize}

例如:
\begin{lstlisting}
template<typename T>
void f(T&& param); //param现在是一个通用引用

int x = 27; //x是int类型
const int cx = x; //cx是const int类型
const int& rx = x; //rx是x的常量引用

f(x); //x是一个左值表达式,因此T是一个int&,param也是int&
f(px); //px是一个左值表达式,因此T是const int,param是const int &
f(rx); //rx是一个左值表达式,因此T是const int,param是const int &
f(27); //27是一个右值表达式,因此T是一个int,param是int &&
\end{lstlisting}

条款24清楚的解释了这些情形发生的原因。关键点在于通用引用的类型推导规则取决于形参是左值引用还是右值引用。当使用通用引用时,类型推导规则会区分左值实参和右值实参,而这从来都不会发生在非通用引用上。

(译者注:以上对左值引用、右值引用、通用引用不太明白的部分可以看\\http://www.cnblogs.com/qicosmos/p/3369940.html学习)

\subsection*{情形3:\textit{ParamType}既不是指针,也不是引用}
当ParamType既不是一个指针,也不是一个引用时,我们使用值传递的方式:
\begin{lstlisting}
template<typename T>
void f(T param); //param通过值传递
\end{lstlisting}
这意味着param将会成为传递进来的实参的一份拷贝---一个全新的对象。事实上,param成为一个全新的对象指导T如何从expr中推导。

1. 与之前一样,如果expr是一个引用,引用的部分会被忽略。
2. 在忽略了expr的引用部分后,如果expr是一个const,同样也会被忽略。如果它是volatile,也会被忽略。(volatile对象并不常用,它们通常仅用于设备驱动程序的开发。详细内容可见条款40。)

因此:
\begin{lstlisting}
int x = 27; //x是int类型
const int cx = x; //cx是const int类型
const int& rx = x; //rx是x的常量引用

f(x); //T和paramdoiu是int
f(cx); //T和param都是int
f(rx); //T和param都是int
\end{lstlisting}

注意尽管cx和rx都是静态变量,但param不是。这是讲得通的,因为param是一个完全独立于cx和rx的对象---它们之一的一份拷贝。事实上,cx和rx不可被修改不会对param产生任何影响。这也是为什么说expr的静态性在类型推导时被忽略:因为expr不可被修改并不意味着它的拷贝也不可被修改。

意识到当使用值传递方式时,const(volatile)会被忽略是很重要的。我们看到,对于形参是静态量引用或指针,在类型推导时expr的静态性被保留。但是当expr是一个指向const对象的const指针,而且expr是通过值传递的方式时:

\begin{lstlisting}
template<typename T>
void f(T param); //param仍通过值传递

const char* const ptr = "Fun with pointers"; //ptr是一个指向const对象的const指针

f(ptr); //实参类型是const char *const
\end{lstlisting}

在这里,最右边的const表明了指针是静态的:意味着ptr不能再次指向别的位置,亦或是空。(左边的const表明ptr所指向的字符串不可能被修改。)当ptr传递给f时,指针按位拷贝到param中。同样的,指针自身也是通过值传递。按照使用值传递方式的类型推导规则,ptr的静态性将被忽略,param的类型推导为const char*,一个指向不可被修改的字符串的指针。在类型推导的过程中,ptr所指向内容的静态性被保留,但是ptr自身的静态性在拷贝并创建新指针的过程中丢失了。

\subsection*{数组实参}
上面的内容已经涵盖了模板类型推导的主流部分,但还有一些少见的情况值得我们了解。数组类型何指针类型是不同的,尽管一些时候它们是可以交换的。这个谬误主要来源于:在很多语境中,一个数组会退化成其中第一个元素的指针。这种退化允许下面的代码能够通过编译:
\begin{lstlisting}
const char name[] = "J.P.Briggs"; //name的类型是const char[13]

const char *prtToName = name; //数组退化成指针
\end{lstlisting}

在这里,const char*的指针ptrToName使用name初始化,这2个类型(const char *和const char[13])是不同的,但是因为数组-指针退化规则,这段代码可以通过编译。

但当一个数组使用值传递方式传给一个模板呢?那会发生什么?

\begin{lstlisting}
template<typename T>
void f(T param); //param通过值传递

f(name); //此时T和param会被推导为什么类型?
\end{lstlisting}

一开始我们观察到,函数没有使用数组作为一个形参的,但是下面的语法确实合法的:

\begin{lstlisting}
void myFunc(int param[]);
\end{lstlisting}

但是在这里数组声明被视作一个指针的声明,一位置myFunc的声明实际上等价于:

\begin{lstlisting}
void myFunc(int *param); //与上面相同的函数
\end{lstlisting}

这种数组与指针形参的等价从C中传递到了C++,产生了数组和指针是相同的这种错觉。

因为数组形参声明被视作指针形参,当数组以值传递方式入一个函数模板时,其被推导为指针类型。这一位置当调用模板f时,它的类型形参T会被推导为const char *:

\begin{lstlisting}
f(name); //name是一个数组,但是T推导为const char*
\end{lstlisting}

这样现在就出现了一个诡计(curve ball)。尽管函数不能真正的声明其参数是数组类型的,但是它们能够声明参数是数组的引用!这样如果我们修改模板f,令其通过引用接收实参,

\begin{lstlisting}
template<typename T>
void f(T& param); //param通过引用传递
\end{lstlisting}

然后我们传入一个数组,

\begin{lstlisting}
f(name); //传递数组给f
\end{lstlisting}

T的实际类型被推导为一个数组!这种类型将会包括数组的大小,在这个例子中,T的类型是const char[13],而且f的形参类型是const char (\&)[13]。是的,这个语法看起来是有毒的,但是知道这个会让你获得其他人所得不到的罕见分数(一些英文幽默)。

有趣的是,声明一个数组的引用能够创建一个能推导出数组元素个数的模板:

\begin{lstlisting}
//在编译时期间返回数组的大小
template<typename T, std::size_t N>
constexpr std::size_t arraySize(T (&)[N]) noexcept
{
	return N;
}
\end{lstlisting}

如条款15所描述的,声明一个函数是constexpr的将使其结果在编译期间可用。这让这种用法变得可能:声明一个与之前创建的数组一样大的数组。

\begin{lstlisting}
int keyVals[] = {1, 3, 7, 9, 11, 22, 35} //keyVals包括7个元素

int mappedVals[arraySize(coevals)]; //mappedVals也有7个元素。
\end{lstlisting}

当然,作为一个现代C++的开发人员,你当然可以用std::array构造一个数组:

\begin{lstlisting}
std::array<int, arraySize(keyVals)> mappedVals; //mappedVals的大小是7
\end{lstlisting}

至于被声明为noexcept的函数arraySize,这将帮助编译器生成更好的代码,详见条款14。

\subsection*{函数实参}
数组类型不是C++中唯一能被退化为指针的类型,函数类型也能够退化为指针类型,我们讨论的任何一个关于类型推导的规则和对数组相关的情形对于函数的类型推导也适用,函数类型会退化为函数的指针,因此:

\begin{lstlisting}
void someFunc(int, double) //一个类型是void(int, double)的函数

template<typename T>
void f1(T param); //值传递方式

template<typename T>
void f2(T& param); //引用传递方式

f1(someFunc); //param被推导为一个函数指针,类型是void(*)(int, double)

f1(someFunc); //param被推导为一个函数引用,类型是void(&)(int, double)
\end{lstlisting}

在实际情况中很难有什么区别,但是当你了解到了数组-指针的退化时,你也同样应该了解到函数-指针的退化。

所以你现在明白了:auto相关的模板类型推导规则。如我所说,在一开始它们大多都非常简单直接。唯一需要特殊处理的只有在使用通用引用时,左值表达式会有特别的处理。然而数组和函数的退化规则使得规则变得更加混乱。有时,你可能只是简单的抓住你的编译器,”告诉我,你推导出的类型是什么“,这时候,你可以看看条款4,因为条款4就是讲述如何劝诱你的编译器这么做的。

\begin{mdframed}
请记住:
\begin{itemize}
\item{在模板类型推导时,对引用实参会被视作非引用类型的。} 
\item{当推导通用引用的形参,左值实参需要特殊的处理。}
\item{当推导使用值传递方式的形参时,const和volatile类型的实参会被忽略掉这两个属性。}
\item{在模板类型推导时,数组和函数实参中会退化为指针类型,除非它们被实例化为引用。}
\end{itemize}
\end{mdframed}

\section{条款2:理解auto类型推导}

如果你已经阅读了条款1,那么你几乎已经掌握了关于auto类型推导的全部知识,因为除了一个例外之外,auto类型推导几乎就是模板类型推导。但是怎么会呢?模板类型推导包括了模板、函数和形参,但是auto并不处理它们中的任一个。

事实确实如此,但是也并没有关系。模板类型推导和auto类型推导之间存在一个映射。通过一种逐字逐句的算法进行互相转换。

在条款1中,模板类型推导使用了如下的函数模板:

\begin{lstlisting}
template<typename T>
void f(ParamType param);
\end{lstlisting}

和如下的函数调用:

\begin{lstlisting}
f(expr); //使用表达式调用f
\end{lstlisting}

在f的函数调用中,编译器使用expr去推导T和ParamType的类型。当一个变量使用auto声明时,auto扮演了模板中T的角色,变量的类型说明符(specifier)则扮演了ParamType的角色。这里用以下的例子能够更好的描述这个情形:

\begin{lstlisting}
auto x = 27;
\end{lstlisting}

这里x的类型说明符就是auto本身。从另一方面来讲,在这个声明中,

\begin{lstlisting}
const auto cx = x;
\end{lstlisting}

的类型说明符是const auto。并且在这里,

\begin{lstlisting}
const auto& rx = x;
\end{lstlisting}

的类型描述符是const auto\&。为了推导例子中x,cx和rx的类型,编译器会认为每个声明都是一个模板,并且按照模板的方式来初始化表达式:

\begin{lstlisting}
template<typename T>
void func_for_x(T param); //推导x类型的概念模板(conceptual template)

func_for_x(27); //概念调用:param的推导类型是x的类型

template<typename T>
void func_for_x(const T param); //推导cx类型的概念模板

func_for_cx(x); //概念调用:param的推导类型是cx的类型

template<typename T>
void func_for_rx(const T& param); //推导rx类型的概念模板

func_for_rx(x); //概念调用:param的推导类型是rx的类型
\end{lstlisting}

如我所说,除了一个例外外,auto的类型推导与模板的类型推导规则一样。

条款1基于ParamType的特点、函数模板中param的类型描述符,将模板类型的推导分为3种情形。在使用auto的变量推导中,类型描述符取代了ParamType的位置,因此这里同样也有3种情形:

\begin{itemize}
\item{情形1:类型描述符是一个指针或引用,但不是一个通用引用。}
\item{情形2:类型描述符是一个通用引用。}
\item{情形3:类型描述符既不是一个指针,也不是一个引用。} 
\end{itemize}

我们已经看到了情形1和情形3:

\begin{lstlisting}
auto x = 27; //情形3,x不是指针,也不是引用。

const auto cx = x; //情形3,cx也不是。

const auto& rx = x; //情形1,rx是一个非通用引用。
\end{lstlisting}

情形2与你期望的一样:

\begin{lstlisting}
auto&& uref1 = x; //x是一个int类型的左值表达式,所以uref1是int&

auto&& uref2 = cx; //cx是一个const int类型的左值表达式,所以uref2是const int &

auto&& uref3 = 27; //27是一个int类型的右值表达式,所以uref3是int&&
\end{lstlisting}

条款1总结了数组和函数如何退化为非引用类型指针的类型描述符。这种情形也在auto类型推断中发生了:

\begin{lstlisting}
const char name[] = "R. N. Briggs"; //name的类型是const char[13]

auto arr1 = name; //arr1的类型是const char *

auto& arr2 = name; //arr2的类型是const char &[13]

void someFunc(int, double); //someFunc是一个void(int, double)类型的函数

auto func1 = someFunc; //func1的类型是void(*)(int, double)

auto& func2 = someFunc; //func2的类型是void(&)(int, double)
\end{lstlisting}

如你所见,auto类型推导与模板类型推导非常相似。它们就像是硬币的两边。

除了一个例外。我们将会通过一个例子开始:你希望声明一个初值为27的int对象,C++98给了2种方法:

\begin{lstlisting}
int x1 = 27;
int x2(27);
\end{lstlisting}

通过C++11中的同意初始化(Uniform Initialzation),加入了这些方法:

\begin{lstlisting}
int x3 = {27};
int x4{27};
\end{lstlisting}

总共使用了4种语法,获得了同样的结果:一个初值为27的int对象。

但如条款5中描述的那样,使用auto类型相较于使用固定类型声明变量有许多好处。因此使用auto替换上述例子中的int将会非常愉快的。简单的文本替换后,变成了如下代码:

\begin{lstlisting}
auto x1 = 27;
auto x2(27);
auto x3 = { 27 };
auto x4{ 27 };
\end{lstlisting}

这些声明都能够通过编译,但是它们却并不代表相同的含义。前两个声明确实使用27声明了1个int类型的变量。然而后两个却是声明了一个拥有一个元素27的std::initializer\_list<int>类型的变量!

\begin{lstlisting}
auto x1 = 27; //类型是int,值是27
auto x2(27); //同上
auto x3 = { 27 }; //类型是std::initializer_list<int>,值是{27}
auto x4{ 27 }; //同上
\end{lstlisting}

这种情形是源于auto的特殊类型推导规则。当一个auto声明的变量的初始化器(Initializer)放置在一对大括号中时,这个变量的推导类型是的std::initializer\_list。如果这个类型不能被推导(例如大括号中的值都是不同类型的),这段代码将会无法通过编译。

\begin{lstlisting}
auto x5 = {1, 2, 3.0}; //错误!不能为std::initializer_list<T>推导T
\end{lstlisting}

如注释所指出的那样,在这种情形中,类型推导失败了。但是理解在这个地方实际上发生了两种类型的类型推导是很重要的。一种源于auto的使用:x5的类型必须被推导。因为x5的初始化器在大括号中,所以x5必须被推断为std::initializer\_list。第二种则是,因为std::initializer\_list是一个模板,std::initializer\_list<T>的为某种类型的T实例化,这也意味着T的类型必须被推导。上述的类型推导是因为第二种:模板类型推导而失败的,因为大括号中的初始化器中的元素有多种类型。

对于大括号初始化器的不同的处理方式是auto类型推导和模板类型推导唯一不同的地方。当auto声明的变量被大括号初始化器初始化时,推导出的类型是std::initializer\_list。但是如果相应的模板被传入一个相同的初始化器时,类型推导会失败,代码无法通过编译。

\begin{lstlisting}
auto x = {11, 23, 9} //x的类型是std::initializer_list<int>

template<typename T>
void f(T param); //与x声明等价的模板形参声明

f({11, 23, 9}); //错误,无法为T推导类型
\end{lstlisting}

然而,当你指定形参的类型是std::initializer\_list<T>时,模板类型推导规则就会成功推导出T的类型:

\begin{lstlisting}
template<typename T>
void f(std::initializer_list<T> initList);

f({11, 23, 9}); 
//T被推导为int,initList的类型是std::initializer_list<int>
\end{lstlisting}

所以auto和模板类型推导的唯一差别就是:auto假定大括号初始化器代表着std::initializer\_list,而模板类型则不同。

你可能希望知道为什么auto类型推导对于大括号初始化器使用了特别的规则,但是模板类型推导没有。我也想知道,但是不幸的是,我没有找到一个方便的解释。因为规则就是规则,这意味着你在使用auto声明一个变量,并使用一个大括号初始化器时必须记住:推导出的类型一定是std::initializer\_list。如果你想更深入的使用统一的集合初始化时,你就更要牢记这一点。C++11中一个最经典的错误就是程序员意外的声明了一个std::initializer\_list类型的变量,但他们的本意却是想声明一个其他类型的变量。错误造成的主要原因是一些程序员只有当必要的时候,才使用大括号的初始化式进行初始化。(将会在条款7中详细讨论。)

对于C++11来说,这已经是一个完整的故事了,但是对于C++14,故事还没有结束,C++14允许auto来表示一个函数的返回值的类型(见条款3),并且C++14的lambda表达式可以在参数的声明时使用auto。不管怎样,这些auto的使用,采用的都是模板类型推导的规则,而不是auto类型推导规则,这意味着,大括号的初始化式会造成类型推导的失败,所以一个带有auto返回类型的函数如果返回一个大括号的初始化式将不会通过编译。

\begin{lstlisting}
auto createInitList() 
{
	return {1, 2, 3}; //错误,不能够推导类型f{1, 2, 3}
}
\end{lstlisting}

同样,规则也适用于当auto用于C++14的lambda的参数类型说明符时:

\begin{lstlisting}
std::vector<int> v;
...

auto resetV = 
	[&V](const auto& newValue) { v = newValue; }; //C++14
...

resetV( {1, 2, 3} ); //错误,不能够推导类型f{1, 2, 3}
\end{lstlisting}

\begin{mdframed}
请记住:
\begin{itemize}
\item{auto类型推导规则通常与模板类型推导相同,但是auto类型推导假定1个大括号初始化器代表着std::initializer\_list,而模板类型推导不然。}
\item{当auto是一个函数的返回值类型或是一个lambda传递形参类型时,使用模板类型推导规则,而不是auto类型推导规则。} 
\end{itemize}
\end{mdframed}

\section{条款3:理解decltype}

decltype是一个古怪的东西。给定一个名称或者表达式,decltype能告诉你它们的类型。通常用来告诉你它们的类型是不是你想要的。然而有的时候,它也会让你百思不得其解,转而向在线的Q\%A网站求助。

我们将从典型的案例开始,它们的结果通常在你的意料之中。与模板类型推导和auto类型推导不同,decltype会返回你给出的名称和表达式准确的类型:

\begin{lstlisting}
const int i = 0; //decltype(i)返回const int

bool f(const Widget& w); //decltype(w)返回const Widget&
                         //decltype(f)返回bool(const Widget &)
                         
struct Point {
	int x, y;
}; 
//decltype(Point::x)返回int
//decltype(Point::y)返回int

Widget w; //decltype(w)返回Widget

if (f(w)) ...  //decltype(f(w))返回bool

template<typename T>       // std::vector的简单版本 
class vector { 
public: 
	… 
	T& operator[](std::size_t index); 
	… 
}; 

vector<int> v;            // decltype(v)返回vector<int> 
… 
if (v[0] == 0) …          // decltype(v[i])返回int&

\end{lstlisting}

看到了吗?并没有什么令人惊讶的。

在C++11中,可能decltype的主要用处是声明函数模板,当其的返回类型取决于参数类型时。举个例子,假定我们要写一个函数,它的参数是一个支持[ ]下标访问的容器,函数首先对使用者进行严验证,然后返回下标操作的结果。函数返回值的类型应该与下标操作返回值的类型相同。

对一个对象类型为T的容器使用[ ]运算符应当返回一个T\&类型的对象,std::deque就是这样。std::vector几乎也是这样,但只有一个例外,对于std::vector<bool>,[ ]运算符并不返回一个bool\&类型的对象,而是返回一个全新的对象,条款6会解释这样的原因。但是重要的是,作用在容器上的[ ]运算符的返回类型取决于这个容器本身。

decltype让这件事变得简单。下面是我们写的第一个版本,显示了使用decltype推导返回类型的方法,这个模板还可以更精简一些,但是我们先暂时不考虑这个:

\begin{lstlisting}
template<typename Container, typename Index> //可以工作 
auto authAndAccess(Container& c, Index i)    //但是能再精简一些
	-> decltype(c[i])                            
{ 
	authenticateUser(); 
	return c[i]; 
}
\end{lstlisting}

在函数名前使用auto不会进行任何的类型推导,它暗示了C++11的返回类型追踪(trailing return type)语意正在使用。例如:函数的返回类型将在参数列表后声明(在->后面)。追踪返回类型的好处是函数的参数能够在能在返回类型的声明中使用。例如在authAndAccess中,我们使用c和i表明函数的返回类型。如果我们想要将返回类型声明在函数名的前面,但是此时c和i是不可用的,因为它们此时还没有声明。

使用例子中的声明方法,authAndAccess能够返回[ ]运算符所返回的类型,如我们想要的一样。

C++11允许推导单一lambda语句的返回类型,C++14扩展了这个功能,使得所有的lambda和函数表达式都能够使用,包括含有多条语句的函数。这意味着,在C++14中,我们可以不需要返回类型追踪,只需要使用一个auto。在这种形式的声明中,auto确实代表着这里应当表达的类型。这意味着编译器将依据函数的内容来推导函数的返回值类型。

\begin{lstlisting}
template<typename Container, typename Index> //c++14支持
auto authAndAccess(Container& c, Index i)    //但不是十分正确
{ 
	authenticateUser(); 
	return c[i]; //由c[i]推断返回类型
}
\end{lstlisting}

条款2描述了auto如何推导函数的返回值类型:编译器使用模板类型推导的规则。在这个例子中,就出现了问题。如我们前面所讨论的,[ ]运算符为大多数T类型的容器返回一个T的引用,但是条款1中又说了:在模板类型推导的过程中,引用性在表达式初始化过程中会被忽略。思考一下下面的代码:

\begin{lstlisting}
std::deque<int> d;
...
authAndAccess(d, 5) = 10; //函数返回d[5]
                          //并为其赋值10
                          //但是不会通过编译!
\end{lstlisting}
在这里,d[5]应当返回一个int\&,但是auto推导的返回类型会忽略掉引用,因此这里的返回值类型是int。这里的int作为一个函数的返回值,是一个右值表达式,而上面的代码尝试将10赋值给一个右值表达式。这在C++中是禁止的,所以这段代码不能通过编译。

为了让authAndAccess能像我们想要的方式工作,我们需要使用decltype为返回值作类型推导,例如:令authAndAccess的返回值类型正好是c[i]表达式所返回的。C++标准的制定者预料到了在某种情况下,类型推导需要使用decltype。所以在C++14中出现了decltype(auto)说明符。刚遇到这种情况时,似乎有一些矛盾。但事实上这是合情合理的,auto指明了了类型需要被推导,而decltype则指示了在推导中所需要使用的规则。因此我们可以这样改写autoAndAccess:

\begin{lstlisting}
template<typename Container, typename Index> //c++14支持
decltype(auto)                               //但还能够改进
authAndAccess(Container& c, Index i)    
{ 
	authenticateUser(); 
	return c[i];
}
\end{lstlisting}

现在authAndAccess的返回值类型将会和c[i]的的返回值完全一致。当c[i]返回一个T\&时,autoAndAccess也会返回一个T\&,而当c[i]返回一个对象时,antuAndAccess也会返回一个对象。

decltype(auto)的用途并不限于函数的返回值类型。当你想要使用decltype类型推导初始化表达式时,它们也能很方便声明变量:

\begin{lstlisting}
Widget w; 
const Widget& cw = w; 
auto myWidget1 = cw;           // auto推导出的: 
                               // myWidget1类型是Widget 
decltype(auto) myWidget2 = cw; // decltype推导出的: 
                               // myWidget2类型是 
                               // const Widget&
\end{lstlisting}

但是我知道这里有两个问题正困扰着你。一个是我之前提到的authAndAccess的改进,让我们现在来解决这个问题。

再看一下C++14版的authAndAccess的声明:
\begin{lstlisting}
template<typename Container, typename Index>
decltype(auto) anthAndAccess(Container& c, Index i);
\end{lstlisting}

其中容器形参是通过非const的左值引用传递,因为对一个容器的引用允许我们修改容器其中的元素。但这也意味着不能够向这个函数传递一个右值容器。右值不能够绑定在一个左值引用上(除非是一个const的左值引用,但本例中不是这样的)。

诚然,向authAndAccess传递一个右值容器是一个特殊情况。一个右值引用一般是一个临时对象,会在调用authAndAccess的函数的语句后摧毁,这也意味着对该容器的某一个元素的引用将会在调用语句的结束时(一般是authAndAccess返回时)悬空。但是,向anthAndAccess传递一个临时变量仍然是有意义的。一个客户可能只是想要拷贝这个临时变量中的一个元素:

\begin{lstlisting}
std::deque<std::string> makeStringDeque(); // 工厂函数 
                                           //从makeStringDeque的函数值中拷贝 
                                           //容器的第五个元素 
auto s = authAndAccess(makeStringDeque(), 5);
\end{lstlisting}

支持这种用法意味着我们需要修改authAndAccess的声明,让其既可以接受左值也可以接受右值。这里可以使用重载(一个重载函数声明一个左值引用形参,另一个重载函数声明一个右值引用形参),但是这样我们就要维护两个函数。一种避免这种情况的方法是令authAndAccess使用一个能绑定左值和右值的引用形参,条款24中阐述了这也正好是通用引用所能做的。因此authAndAccess能像这样声明:

\begin{lstlisting}
template<typename Container, typename Index>
decltype(auto) anthAndAccess(Container&& c, Index i); //c是一个通用引用
\end{lstlisting}

在这个模板里,我们并不知道操作的容器类型,这也意味着我们一样不知道下标所对应对象的类型。对一个为止类型的对象使用传值方法往往会因为不必要的拷贝开销而影响性能,对象分割问题(见条款17)和来自同事的嘲笑。但是根据标准库中的例子(例如std::string, std::vector和std::deque),这种情况下看起来也是合理的,所以我们坚持按值传递。

然而,我们需要更新模板的实现方式,依据条款25的警告,将std::forward应用到通用引用上:

\begin{lstlisting}
template<typename Container, typename Index>
decltype(auto)
anthAndAccess(Container&& c, Index i); //C++14的最终版本
{
	authenticateUser();
	return std::forward<Container>(c)[i];
}
\end{lstlisting}

这样就满足了我们所需要的所有要求,但是这段代码需要C++14的编译器。如果你还没有的话,你需要将其改成C++11的版本。这和C++14版本相似,除了你需要自己标注出返回的类型。

\begin{lstlisting}
template<typename Container, typename Index> //C++11的 
auto                                         // 的最终 
authAndAccess(Container&& c, Index i)        // 版本
-> decltype(std::forward< Container>(c)[i]) 
{ 
	authenticateUser(); 
	return std::forward<Container>(c)[i]; 
}
\end{lstlisting}

另一个值得对你唠叨的问题我已经标注在了这一条款的开始处了,decltype获得的结果几乎和你期待的一样,这并不奇怪。老实说,你几乎不太可能遇到这个规则的例外情况,除非你需要实现一个任务非常繁重的代码库。

为了完全理解decltype的行为,你需要让你自己熟悉一些特殊的情况,大多数在这本书里证明讨论起来会非常的晦涩,但是其中一条能让我们更加理解decltype的使用。

对一个变量名使用decltype产生声明这个变量时的类型。有名字的是左值表达式,但这并不影响decltype的行为。因为对于比变量名更复杂的左值表达式,decltype确保推导出的类型总是一个左值的引用,这意味着如果一个左值表达式不同于变量名的类型T,decltype推导出的类型将会是T\&,这几乎不会照成什么影响,因为大多数左值表达式的类型内部通常包含了一个左值引用的限定符,例如,返回左值的函数总是返回一个左值引用。

这里有一个值得注意的地方:

\begin{lstlisting}
int x = 0;
\end{lstlisting}

x是一个变量的名字,所以decltype(x)的结果是int,但是将名字x用括号包裹起来,”(x)”产生了一个比名字更复杂的表达式,作为一个变量名,x是一个左值,C++同时定义了(x)也是一个左值,因此decltype((x))结果是int\&,将一个变量用括号包裹起来改变了decltype最初的结果!

在C++11中,这仅仅会会让人有些奇怪,但是结合C++14中对decltype(auto)的支持后,你对返回语句的一些简单的修改会影响到函数最终推导出的结果。

\begin{lstlisting}
decltype( auto) f1() 
{ 
	int x = 0;
	… 
	return x;   // decltype(x) 是 int, 所以f1返回int 
} 
decltype(auto) f2() 
{ 
	int x = 0; 
	… 
	return (x); // decltype((x)) 是int&, 所以f2返回int& 
}
\end{lstlisting}

注意到f2和f1不仅仅是返回类型上的不同,f2返回的是一个局部变量的引用!这种代码的结果是未定义的,你当然不希望发生这种情况。

你需要记住的是当你使用decltype(auto)的时候,需要格外注意。一些看起来无关紧要的细节会影响到decltype(auto)推导出的结果,为了确保被推导出的类型是你期待的,可以使用条款4中描述的技术。

但同时不要失去对大局的注意,decltype(无论是独立使用还是和auto一起使用)推导的结果可能偶尔让人惊讶,但是这并不会经常发生。通常,decltype的结果和你所期待的类型一样,尤其是当decltype应用在变量名的时候,因为在这种情况下,decltype做的就是提供变量的声明类型。

\begin{mdframed}
请记住:
\begin{itemize}
\item{decltype几乎总是返回变量名或是表达式的类型而不会进行任何的修改。}
\item{对于不同于变量名的左值表达式,decltype的结果总是T\&。}
\item{C++14提供了decltype(auto)的支持,比如auto,从它的初始化式中推导类型,但使用decltype的推导规则。} 
\end{itemize}
\end{mdframed}










































\section{条款4:掌握如何查看推导类型}

如何选择合适的、用于查看类型推导结果的工具,取决于你在软件开发过程中所处于的阶段。我们将探讨其中的三种可能性:第一种是在编辑代码时获取类型推导信息,第二种是编译代码时获取,最后一种是运行时过程中获取。

\subsection{IDE编辑器}
当你将你的光标放置在实体的附件时,IDE中的编辑器往往能够标注出代码中实体的类型(例如:变量,形参,函数等)。以下面的代码为例;

\begin{lstlisting}
const int theAnswer = 42;

auto x = theAnswer;
auto y = &theAndwer;
\end{lstlisting}

一个IDE的编辑器将会显示x的推导类型是int,而y的推导类型是const int*。

此时此刻,你的代码实际上也处于某种已编译的状态,因为这样IDE才能够通过编译器获取这些信息。如果编译器不能够在此时分析得到足够的类型推导信息,那么IDE也不能够告诉你类型推导的结果。

对于像int这样的简单类型,IDE通常能够获得正确的结果。然而,我们很快就会发现,当使用了复杂类型时,IDE所显示的结果可能就不会有什么作用了。

\subsection{编译器诊断}

利用某种导致编译错误的手段,是一个高效的使用编译器获取类型推导结果的方法。问题的错误提示信息实际上就会暗示是什么类型所导致的。

举个例子,我们想要推导前面例子中x和y的类型。首先声明一个我们并没有定义的模板类,就像这样:

\begin{lstlisting}
template<typename T> //仅为TD声明
class TD;	           //TD的意思是类型显示器
\end{lstlisting}

当尝试实例化这个模板时,编译器会引发一个错误。因为现在并没有用于实例化的模板定义。为了查看x和y的类型,于是使用它们的类型实例化TD:

\begin{lstlisting}
TD<decltype(x)> xType; //显示的错误将会包括x和y的类型
TD<decltype(y)> yType;
\end{lstlisting}

我使用了形如vaiableNameType这样的变量名,因为这样所引出的错误消息将引导我找到想要获取的的信息。对于上面的代码,我的一种编译器生成了如下的诊断信息:

\begin{lstlisting}
error: aggregate 'TD<int> xType' has incomplete type and cannot be defined
error: aggregate 'TD<const int *> yType' has incomplete type and cannot be defined
\end{lstlisting}

另一种编译器使用另一种形式显示了同样的信息:

\begin{lstlisting}
error: 'xType' use undefined class 'TD<int>'
error: 'yType' use undefined class 'TD<const int *>'
\end{lstlisting}

仅仅只是不同的格式。通过这种技术,我所测试的所有编译器都产生我所需要的类型信息。

\subsection{运行时输出}
使用printf手段显示类型的信息只有在运行时才能体现出来(尽管我并不推荐你使用printf),但是这种手段能够完整地控制输出格式。使用文本表示的唯一困难就是如何合适的显示。你一定会觉得,“这有什么难的,使用typeid和std::type\_info::name就能够轻松解决了”。为了能够查看x和y的类型推导,你可能会写出下面的代码:

\begin{lstlisting}
std::cout << typeid(x).name << '\n'; //显示x和y的类型
std::cout << typeid(y).name << '\n';
\end{lstlisting}

这个方法依赖于这样一个事实:对对象x和y使用typeid函数将会返回一个std::type\_info对象,而std::type\_info对象有一个成员函数name,将会产生一个C-style的字符串(如const char *)用于表示类型的名字。

调用std::type\_info::name并不保证返回正确的结果,但是一定程度上会有所帮助。其帮助的等级也会有所不同。例如,GNU和Clang的编译器汇报x的类型是“i”,y的类型是“PKi”。一旦你理解了这些,你就会明白这些输出是有意义的。“i”表示“int”,而“PK”表示“pointer to const”(两个编译器都支持c++filt这个工具,用于解码这些“损坏(mangled)”类型)。微软的编译器产生了更少的输出:“int”表示x,而“int const *”表示y。

这些对于x和y的推导结果都是正确的,所以你可能会认为类型汇报的问题已经解决了。但是请先别忙,考虑一个更加复杂的情况:

\begin{lstlisting}
template<typename T> //被调用的模板函数
void f(const T& param);

std::vector<Widget> createVec(); //工厂函数

const auto vw = createVec(); //使用工厂函数初始化vw

if (!vw.empty()) {
	f(&vw[0]);	//调用f
	...
}
\end{lstlisting}

这段代码中有一个用户定义类型(Widget),一个STL容器(std::vector),和一个auto变量(vw)。这是一个非常典型的情况,你可能会很想知道你的编译器类型推导的结果。例如,模板类型形参T和f的函数形参。

了解问题中得typeid是很容易的。向f函数加入一些代码,你就会看到:

\begin{lstlisting}
template<typename T>
void f(const T& param)
{
	using std::cout;
	cont << "T = 		" << typeid(T).name() << '\n'; //显示T的类型
	cout << "param = 	" << typeid(param).name() << '\n'; //显示param的类型
}
\end{lstlisting}

执行GNU和Clang编译出的代码后输出:

\begin{lstlisting}
	T =		PK6Widget
	param =	PK6Widget
\end{lstlisting}

我们现在已经知道,对于这些编译器而言,“PK”代表“pointer to const”,所以唯一神秘的地方就是数字6。这个简单的代表了接下来类名的字符个数。所以编译器告诉我们T和param都是const Widget*类型的。

微软的编译器也同意:

\begin{lstlisting}
	T =		class Widget const *
	param =	class Widget const *
\end{lstlisting}

3个独立的编译器生成了相同的结果。但是仔细看一下,在模板f中,param的推导类型是const T\&。在这种情况下,T和param拥有相同的类型不是非常奇怪吗?如果T是一个int类型,那么param应该是const int\&,根本不应该是相同的类型。

不幸的是,std::type\_info::name的结果并不可靠。在这个例子中,这3个编译器所汇报param类型都是错误的。而且,它们实际上是被要求输出错误的结果,因为std::type\_info::name明确规定了形参类型推导应当被视作如它们作为一个传值传递的参数一样。正如条款1所说的,这意味着如果类型是一个引用,其引用性会被忽略,如果被去掉引用后是静态类型(或volatile)的,那么其静态性(volatileness)也会被忽略。所以这也是为什么param的类型--const Widget * const \&被汇报为const Widget *。首先引用性被忽略,然后指针的静态性也被消除。

同样不幸的是,IDE编辑器中显示的类型信息也是不可靠的,或者至少是不那么可用的。还是刚才那个例子,一个我所知道的IDE编辑器对T的上报结果如下:

\begin{lstlisting}
const
std::_Simple_types<std::_Wrap_alloc<std::_Vec_base_types<Widget,
std::allocator<Widget> >::_Alloc>::value_type>::value_type *
\end{lstlisting}

同样的编辑器显示param的类型是:

\begin{lstlisting}
const std::_Simple_types<...>::value_type *const &
\end{lstlisting}

比起T类型的类型还是简单一些,但是其中的“...”可能会令你迷惑,直到你意识到了,这是编辑器在告诉你:“我删除了T类型的部分”。运气好的话,你的开发环境应该能比这个做得更好。

如果你更倾向于依赖库而不是幸运,当你知道std::type\_info::name和IDE也会出错,而Boost中的TypeIndex库(常被写作Boost.TypeIndex)一定不会出错时,你一定会很高兴。TypeIndex不是C++标准库的一部分,也不是IDE或者类似TD的模板。Boost库是一个跨平台、开源的,其基于一个即使是最多疑的团队也能够接受的协议,这也意味着使用Boost库的可移植性非常接近C++标准库。

以下使用Boost.TypeIndex实现的代码:

\begin{lstlisting}
template<typename T>
void f(const T& param)
{
	using std::cout;
	using boost::typeindex::type_id_with_cvr;
	
	//显示T的类型
	cout << "T = 	"
	     << type_id_with_cvr<T>().pretty_name();
	     << '\n';
	
	//显示param的类型
	cout << "param = "
	     << type_id_with_cvr<decltype(param)>().pretty_name();
	     << '\n';
	...
}
\end{lstlisting}

模板函数boost::typeindex::type\_id\_with\_cvr接收一个类型参数,它不会移除任何const, volatile或者引用修饰符(所以函数名带有一个“with\_cvr”)。其返回结果是一个boost::typeindex::type\_index对象中的成员函数pretty\_name生成的类型字符串。

在这种实现下,重新考虑一下当使用typeid时param输出的错误类型信息:

\begin{lstlisting}
std::vector<Widget> createVec(); //工厂函数

const auto vw = createVec(); //使用工厂函数初始化vw

if (!vw.empty()) {
	f(&vw[0]);	//调用f
	...
}
\end{lstlisting}

使用GNU和Clang的编译器编译后。Boost.TypeIndex生成了这样的(精确的)输出:

\begin{lstlisting}
T =		Widget const*
param =	Widget const* const& 
\end{lstlisting}

使用微软编译器也生成了相同的信息:

\begin{lstlisting}
T =		class Widget const*
param =	class Widget const* const& 
\end{lstlisting}

这种近乎一致的表示是很好的,但是也要记住IDE编辑器,编译错误信息或者类似Boost.TypeIndex的库都仅仅只是帮助你查看编译器的推导信息的辅助工具。它们可能会很有用,但是深刻理解条款1-3所带来的好处是不可替代的。

\begin{mdframed}
请记住:
\begin{itemize}
\item{使用IDE编辑器,编译错误信息和Boost TypeIndex库可以查看类型推导的结果。}
\item{使用某些工具获取的推导结果不一定是精确的,因此理解C++的类型推导规则仍然是非常重要的。}
\end{itemize}
\end{mdframed}























